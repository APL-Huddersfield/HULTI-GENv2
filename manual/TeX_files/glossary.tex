\newglossaryentry{absorb}{
	name={absorption},
	description={The process of absorbing a portion of the energy of a wave as it reflects from a surface, determined by an absorption coefficient}
}

\newglossaryentry{aiff}{
	name={AIFF},
	description={An uncompressed type of digital audio file with a fixed sample rate, developed by Apple Inc}
}


\newglossaryentry{algorithm}{
	name={algorithm},
	description={Describes the steps of a procedure required to accomplish a particular task}
}

\newglossaryentry{algorithmic}{
	name={algorithmic reverberator},
	description={A reverberator that utilises an algorithm of delays and filters in order to simulate the reverberation of a space}
}

\newglossaryentry{ambisonic}{
	name={ambisonic},
	description={A type of audio format that encodes audio signals into multiple directions. An ambisonic microphone, therefore, acheives this using a tightly packed cluster of microphones}
}

\newglossaryentry{anechoic}{
	name={anechoic},
	description={Without echo. An anechoic chamber for example is a room without echoes}
}

\newglossaryentry{apf}{
	name={allpass filter},
	description={Similar to the comb filter, except the signal path is configured so that the filter has a flat frequency response, allowing all frequencies to pass through}
}

\newglossaryentry{asw}{
	name={ASW},
	description={Apparent Source Width. A sub-paradigm of spatial impression that describes the perceived width of an auditory source}
}

\newglossaryentry{aswmax}{
	name={ASW\textsubscript{max}},
	description={A region on the horizontal plane where, in the presence of front arriving direct sound, an early reflection arriving anywhere within the region will produce the maximum degree of perceived ASW}
}

\newglossaryentry{beam}{
	name={beam tracing},
	description={A geometric method that traces sound beams from surface to surface to pre-calculate possible paths sound could take from source to receiver to a given order.}
}

\newglossaryentry{bilin}{
	name={bilinear interpolation},
	description={A form of linear interpolation where the same process is applied in two dimensions between four points}
}

\newglossaryentry{brir}{
	name=BRIR,
	description={Binaural Room Impulse Response. An impulse response of a space measured using a dummy head or rendered using a HRTF database, and designed to represent how a listener would perceive the actual space in real life}
}

\newglossaryentry{comb}{
	name={comb filter},
	description={A type of filter that mixes a delayed signal with the original signal. The name describes the characteristic 'hair comb' shape of the filter's frequency response}
}

\newglossaryentry{convolution}{
	name=convolution,
	description={A mathematical process that applies the characteristics of a signal or impulse response onto another signal. It is commonly used in music production for applying realistic reverb to a track, as well as in spatialisation of an audio signal to simulate spatial audio over headphones}
}

\newglossaryentry{dsb}{
	name=DSB,
	description={Degree of Source Broadening. A combination of the IACC and sound strength measures used to predict the perceived width of an auditory source}
}

\newglossaryentry{diffraction}{
	name=diffraction,
	description={Occurs when the a wave passes around the edge of a surface or an obstacle such as a corner, distorting the path and causing it \emph{bend} around the edge}
}

\newglossaryentry{diffusion}{
	name=diffusion,
	description={The process of randomly scattering the sound wave in different directions by a certain degree upon reflection}
}

\newglossaryentry{dwm}{
	name=DWM,
	description={Digital Waveguide Mesh. A wave based method of modelling the propagation of sound, taking into account the wave effects such as interference and diffraction}
}

\newglossaryentry{fdn}{
	name=FDN,
	description={Feedback Delay Network. A type of algorithmic reverberator that models the diffuse reverberant tail. It achieves this by feeding the output of several parallel delay units into each other via a feedback matrix}
}

\newglossaryentry{fft}{
	name=FFT,
	description={Fast Fourier Transform. An optimised version of the Discrete Fourier Transform (DFT) which converts a signal from the time domain and into its frequency domain resprenstation, allowing for analysis of the signal's frequency response, and also an efficient method of convolution}
}

\newglossaryentry{fir}{
	name=FIR,
	description={Finite Impulse Response. A type of digital filter that has an impulse response with a fixed length}
}

\newglossaryentry{hrtf}{
	name=HRTF,
	description={Head Related Transfer Function. A function that describes how sound arrives at a listener's ears from a particular direction in terms of frequency and phase response}
}

\newglossaryentry{ir}{
	name=IR,
	description={Impulse Response. A signal that describes how a system such as a filter or a reverberant space reacts when excited by an extremely short impulse of energy}
}

\newglossaryentry{iir}{
	name=IIR,
	description={Infinite Impulse Response. A type of digital filter that has an impulse response with an infinite length}
}

\newglossaryentry{ism}{
	name=ISM,
	description={Image Source Method. A geometric method of calculating the exact path sound will traverse from source to listener, taking into account each possible set of reflections the path will take}
}

\newglossaryentry{iacc}{
	name=IACC,
	description={Interaural Cross-correlation Coeffcient. The value and associated point of maximum correlation, or similarity, between two ear signals}
}

\newglossaryentry{iacf}{
	name=IACF,
	description={Interaural Cross-correlation Function. Describes the correlation, or similarity, between two signals at different offsets from each other}
}

\newglossaryentry{ild}{
	name=ILD,
	description={Interaural Level Difference. The level difference between the two ear signals}
}

\newglossaryentry{itd}{
	name=ITD,
	description={Interaural Time Difference. The time difference of when sound arrives at each ear}
}

\newglossaryentry{linterp}{
	name={linear interpolation},
	description={The process of linearly cross-fading or \emph{blending} between two values or points}
}

\newglossaryentry{lev}{
	name=LEV,
	description={Listener Envelopment. A sub-paradigm of spatial impression that describes the feeling being enveloped by the sound. It is generally considered an environment related concept, dependent on late reverberance}
}

\newglossaryentry{lf}{
	name=L\(_f\),
	description={Lateral Fraction. Predicts ASW by calculating the ratio of early lateral reflection energy to total early reflection energy}
}

\newglossaryentry{local}{
	name={localisability},
	description={The ease of being able to localise and determine the location of an auditory source}
}

\newglossaryentry{metadata}{
	name={meta-data},
	description={Information that describes the characteristics of an object or entity. A reflection, for example, is characterised by it's direction of arrival, energy and delay time}
}

\newglossaryentry{octave}{
	name=octave,
	description={A musical interval between two frequencies spaced by a ratio of 2:1}
}

\newglossaryentry{phasiness}{
	name=phasiness,
	description={Describes a type of fluctuating, metallic, comb-filter like quality in the perceived sound}
}

\newglossaryentry{riv}{
	name=RIV,
	description={Raw Impulse Vector. A raw reprensentation of an impulse response that contains each captured sound ray along with meta-data about each ray. This meta-data includes the direction of arrival, distance travelled, reflection order, octave-band energy and reflection history}
}

\newglossaryentry{rayTracing}{
	name={ray tracing},
	description={A geometric method that calculates all of the possible paths sound will traverse around a space by emitting a lage number of rays in all directions from around the source, with a chance that they may hit a receiver} 
}

\newglossaryentry{reverb}{
	name=reverberation,
	description={A term used to characterise how the sound propagates around a space, reflecting off objects and surfaces whilst decaying over time. Is generally shortened to reverb}
}

\newglossaryentry{rir}{
	name=RIR,
	description={Room Impulse Response. An impulse response of a room, describing how and when all of the reflections arrive and decay over time. When an accurate RIR is convolved with an audio signal, the resulting reverb closely resembles the real reverb of the space}
}

\newglossaryentry{ge}{
	name=G\(_E\),
	description={Sound Strength. Measures the ratio of sound energy radiated from a source in a reverberant space relative to the energy radiated from the same source in a free field, or anechoic space, measured at 10m}
}

\newglossaryentry{sr}{
	name={sample rate},
	description={The rate that audio data is digitally captured or reproduced, usually at a frequency that is at least twice that of the average human hearing frequency range}
}

\newglossaryentry{si}{
	name=SI,
	description={Spatial Impression. Describes how a listener will perceive the size of an acoustic space}
}

\newglossaryentry{specular}{
	name=specular,
	description={Describes a perfect reflection who's angle of incidence or arrival is equal to its angle of reflection. It is the opposite of a diffuse reflection}
}

\newglossaryentry{tonalcolour}{
	name={tonal colouration},
	description={A change in the timbre of the sound due to interference between the orignal sound and reflections}
}

\newglossaryentry{vr}{
	name={VR},
	description={Virtual Reality. Simulating reality using a computer}
}

\newglossaryentry{wav}{
	name=WAV,
	description={An uncompressed type of digital audio file with a fixed sample rate, developed by Microsoft and IBM}
}
