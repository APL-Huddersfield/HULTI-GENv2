\chapter{CSV results file format}

\section{Individual subject CSV files}

Once a subject has completed all sessions in a test, HULTI-GENv2 will prompt the user to save a 'Comma Separated Values' (.CSV) file. This file is saved in conjuction with the subject's .JSON file, and is supported by most statistical analysis and speadsheet applications. The data in each column of a CSV file is formatted in hierarchical fashion, depending on the procedure. For grading and non-adaptive psychophysical tests, e.g. 1534-2 MUSHRA, MoC 2AFC etc, the hierarchy is:

\begin{center}
	\textbf{Session $\rightarrow$ Group $\rightarrow$ Stimulus $\rightarrow$ Repetition $\rightarrow$ Response / Grading Value}
\end{center}
\noindent
Whilst for adaptive methods, e.g. Staircase 2AFC, the hierarchy is:

\begin{center}
	\textbf{Session $\rightarrow$ Group $\rightarrow$ Trial $\rightarrow$ Step / Stimulus $\rightarrow$ Response}
\end{center}
\pagebreak
The following table is an example of a subject's CSV file for a 1534-2 MUSHRA, grading test arranged as 2 Sessions, 2 Groups, 3 Stimuli per Group repeated 2 times:

\begin{center}
	\begin{tabularx}{\textwidth}{|X|X|X|X|X|}
		\hline
		\textbf{Session ID} & \textbf{Group ID} & \textbf{Stimulus} & \textbf{Repetition} & \textbf{Response} \\
		\hline
		0 & 0 & 0 & 0 & 100. \\
		0 & 0 & 0 & 1 & 100. \\
		0 & 0 & 1 & 0 & 97. \\
		0 & 0 & 1 & 1 & 98. \\
		0 & 0 & 2 & 0 & 50. \\
		0 & 0 & 2 & 1 & 45. \\
		\hline
		0 & 1 & 0 & 0 & 100. \\
		0 & 1 & 0 & 1 & 100. \\
		0 & 1 & 1 & 0 & 86. \\
		0 & 1 & 1 & 1 & 90. \\
		0 & 1 & 2 & 0 & 31. \\
		0 & 1 & 2 & 1 & 38. \\
		\hline
		1 & 0 & 0 & 0 & 100. \\
		1 & 0 & 0 & 1 & 100. \\
		1 & 0 & 1 & 0 & 90. \\
		1 & 0 & 1 & 1 & 89. \\
		1 & 0 & 2 & 0 & 44. \\
		1 & 0 & 2 & 1 & 40. \\
		\hline
		1 & 1 & 0 & 0 & 100. \\
		1 & 1 & 0 & 1 & 100. \\
		1 & 1 & 1 & 0 & 75. \\
		1 & 1 & 1 & 1 & 78. \\
		1 & 1 & 2 & 0 & 20. \\
		1 & 1 & 2 & 1 & 25. \\
		\hline
	\end{tabularx}
\end{center}

\pagebreak
The following table is an example of a subject's CSV file for an Adaptive Staircase 2AFC test arranged as 2 Sessions and 2 Groups per session (\textit{\textbf{Note:} For clarity, the table has been truncated}):

\begin{center}
	\begin{tabularx}{\textwidth}{|X|X|X|X|X|}
		\hline
		\textbf{Session ID} & \textbf{Group ID} & \textbf{Trial} & \textbf{Step} & \textbf{Response} \\
		\hline
		0 & 0 & 0 & 10 & 1 \\
		0 & 0 & 1 & 10 & 1 \\
		0 & 0 & 2 & 9 & 1 \\
		0 & 0 & 3 & 9 & 1 \\
		0 & 0 & 4 & 8 & 1 \\
		0 & 0 & 5 & 8 & 1 \\
		0 & 0 & 6 & 7 & 1 \\
		0 & 0 & 7 & 7 & 1 \\
		0 & 0 & 8 & 6 & 1 \\
		0 & 0 & 9 & 6 & 1 \\
		0 & 0 & 10 & 5 & 1 \\
		0 & 0 & 11 & 5 & 0 \\
		0 & 0 & 12 & 6 & 0 \\
		0 & 0 & 13 & 7 & 1 \\
		0 & 0 & 14 & 7 & 1 \\
		0 & 0 & 15 & 6 & 1 \\
		0 & 0 & 16 & 6 & 1 \\
		0 & 0 & 17 & 5 & 1 \\
		0 & 0 & 18 & 5 & 1 \\
		\vdots & \vdots & \vdots & \vdots & \vdots\\
		\hline
		0 & 1 & 0 & 10 & 1 \\
		0 & 1 & 1 & 10 & 1 \\
		0 & 1 & 2 & 9 & 1 \\
		0 & 1 & 3 & 9 & 1 \\
		0 & 1 & 4 & 8 & 1 \\
		0 & 1 & 5 & 8 & 1 \\
		0 & 1 & 6 & 7 & 1 \\
		0 & 1 & 7 & 7 & 1 \\
		0 & 1 & 8 & 6 & 1 \\
		0 & 1 & 9 & 6 & 1 \\
		0 & 1 & 10 & 5 & 1 \\
		0 & 1 & 11 & 5 & 0 \\
		0 & 1 & 12 & 6 & 1 \\
		0 & 1 & 13 & 6 & 1 \\
		0 & 1 & 14 & 5 & 1 \\
		0 & 1 & 15 & 5 & 1 \\
		0 & 1 & 16 & 4 & 1 \\
		0 & 1 & 17 & 4 & 1 \\
		0 & 1 & 18 & 3 & 1 \\
		\vdots & \vdots & \vdots & \vdots & \vdots\\
		\hline
	\end{tabularx}
\end{center}
