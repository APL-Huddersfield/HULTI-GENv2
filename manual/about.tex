\chapter{Introduction}
HULTI-GEN is an acronym for 'Huddersfield Universal Listening Test Interface Generator'. HULTI-GEN was originally developed by in 2015 by Chris Gribben, and has now been updated to version 2.0 in 2020 by Dale Johnson, both at the Applied Psychoacoustics Laboratory at The University Of Huddersfield. Its aim is to simplify the design and execution of listening test experiments, and to allow users to modify and customise test interfaces to suit their experimental needs. It is a completely free, standalone, Max-based tool, and \emph{does not} require a Max license in order to function. 
\\
\\
Version 2.0 introduces an overhauled test setup assistant, and a suite of psychophysical test methods such as Two-Alternative Forced-Choice (2AFC), ABX, and Adaptive Staircase. This version is now open-source under the MIT license, which means that the HULTI-GEN program and its built-in tests can be modified within Max, although this does require a Max license in order to save modified patcher files.
\\
\\
Before using HULTI-GEN, the user must have a clear idea of their testing format with consideration toward the number stimuli, the number of sessions that subjects will undertake, and what attributes are being tested. It is therefore recommended that a testing plan is documented before configuring and running an experiment.
\subsubsection{Software requirements}
HULTI-GEN requires Max 8 or higher available from \href{https://cycling74.com/downloads}{https://cycling74.com/downloads}. It will \textbf{NOT} run on earlier versions due to the use of the Multi-Channel (MC) audio features only available since Max 8. A Max license is only required to save any modifications to the source patcher file.
\subsubsection{Important information}
\textbf{Before running HULTI-GEN}, ensure that the \textit{Overdrive} feature of Max is disabled, and that any stimuli files imported that should accompany the test configuration file should be included in the same folder as the HULTI-GEN project file (HULTI-GENv2.maxproj).
\\
\\
The authors do not assume liability for the correctness of the content in this document, and does not take responsibility for any harm or damage resulting from the use of its application. For further information, or would like to give comments or report bugs in HULTI-GEN, do not hesitate and get in touch with us via email on:
\begin{center}
\href{mailto::d.s.johnson2@hud.ac.uk}{Dale Johnson - d.s.johnson2@hud.ac.uk}
\\
\href{mailto::h.lee@hud.ac.uk}{Hyunkook Lee - h.lee@hud.ac.uk}
\end{center}
